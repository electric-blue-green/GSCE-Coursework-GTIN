\documentclass[a4paper]{article}
\usepackage{geometry}
\geometry{a4paper, portrait, margin=1in}
\usepackage[english]{babel}
\usepackage[utf8]{inputenc}
\usepackage{natbib}
\usepackage{graphicx}
\usepackage{fancyhdr}
\usepackage{array}
\usepackage{tabu}
\usepackage{listings}
\usepackage[export]{adjustbox}
\graphicspath{{/Users/ThomasBass/GitHub/GSCE-Coursework-Python-GTIN/Images/}}
\DeclareGraphicsExtensions{.png}

\title{Computing GCSE Coursework}
\author{\\ \\ \\ \\ Thomas Bass\\Candidate 4869\\Centre 52423\\OCR A453 Programming Project\\\\ Word Count 1730 \\\\\ Made with \LaTeX}
\date{2016-2017}


\pagestyle{fancy}
\fancyhf{}
\rhead{Computing GCSE Coursework}
\chead{Candidate 4869}
\lhead{Thomas Bass}
\rfoot{Page \thepage}

\begin{document}

%%test table
\begin{center}
\begin{tabular}{ | m{4em} | m{19em} | m{7em} | m{10em} | }
  \hline
  Test Number & Test Name & Test Data & Test Type	\\ [0.5ex] 
  \hline\hline
  1 & Input strings of incorrect length. If rejected, it passes & \verb|12345| & Error \\
  \hline 
  2 & Input strings of incorrect length. If rejected, it passes & \verb|1234567890| & Error \\
  \hline 
  3 & Input strings of letters. If rejected, it passes. & \verb|abcdefg| & Error \\
  \hline 
  4 & Run a valid input. Print out totals at each stage, and manually check. If the calculations are correct, it passes. & \verb|13245627| & normal \\
  \hline 
  5 & Run the program with a GTIN number taken from a product. If it correctly calculated and verified, it passes.
 & \verb|01412871| & Error \\
  \hline
\end{tabular}
\end{center}

\end{document}
